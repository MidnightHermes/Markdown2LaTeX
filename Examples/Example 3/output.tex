\documentclass{article}
% Package for using images
    \usepackage{graphicx}

% Code to handle lists
    % Ensures nested ordered lists start at 1
    \renewcommand{\theenumii}{\arabic{enumii}}
    \renewcommand{\labelenumii}{\theenumii.}

    % Ensures nested unordered lists are represented by a hollow circle like in Markdown
    \renewcommand{\labelitemii}{$\circ$}

% Remove section numbering as well
    \renewcommand{\thesection}{}
    \renewcommand{\thesubsection}{}
    \renewcommand{\thesubsubsection}{}

% Package and additional code for handling hyperlink formatting
    \usepackage{hyperref}
    \hypersetup{
        colorlinks=true,
        linkcolor=black,
        filecolor=magenta,
        urlcolor=blue,
        pdftitle={Markdown Converted to LaTeX},
        pdfpagemode=FullScreen,
        }
    \urlstyle{same}
    \renewcommand{\UrlFont}{\normalfont\itshape\underline}
    \usepackage[dvipsnames]{xcolor}

% Package for handling block quotes
    \usepackage{csquotes}

\begin{document}

% This removes page numbers from every page,
% Markdown doesn't use them so we don't need them!
    \pagestyle{empty}

\section{My First Markdown File}


\subsection{Introduction}

Markdown is a lightweight markup language with plain text formatting syntax.


\subsection{Features}

\begin{itemize}
    \item Simple and easy to learn
    \item Supports headers, lists, and emphasis
\end{itemize}


\subsection{More Information}

For more details, visit \href{https://www.markdownguide.org/basic$\backslash$-syntax/}{Markdown Guide}.


\includegraphics{https://example.com/sample-image.jpg}


\texttt{Code snippet: echo "Hello, Markdown!"}


\section{Short Nested List}


\begin{itemize}
    \item List
\begin{itemize}
    \item List
\begin{itemize}
    \item List
\end{itemize}
\end{itemize}
\end{itemize}


\section{Longer Nested List}


\begin{itemize}
    \item List
\begin{enumerate}
    \item Hello
    \item World!
\begin{enumerate}
    \item Nest it!
\end{enumerate}
\end{enumerate}
    \item Still a list
\begin{itemize}
    \item Lots of
    \item nested items
\begin{itemize}
    \item and
\end{itemize}
    \item even
    \item more
\end{itemize}
\end{itemize}

\begin{enumerate}
    \item now the list
    \item is ordered
\begin{itemize}
    \item And we can nest
    \item Unordered in ordered
\end{itemize}
    \item or
\begin{enumerate}
    \item ordered in ordered
\end{enumerate}
\end{enumerate}

\end{document}