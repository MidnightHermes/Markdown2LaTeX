\documentclass{article}
% Package for using images
    \usepackage{graphicx}

% Code to handle lists
    % Ensures nested ordered lists start at 1
    \renewcommand{\theenumii}{\arabic{enumii}}
    \renewcommand{\labelenumii}{\theenumii.}

    % Ensures nested unordered lists are represented by a hollow circle like in Markdown
    \renewcommand{\labelitemii}{$\circ$}

% Remove section numbering as well
    \renewcommand{\thesection}{}
    \renewcommand{\thesubsection}{}
    \renewcommand{\thesubsubsection}{}

% Package and additional code for handling hyperlink formatting
    \usepackage{hyperref}
    \hypersetup{
        colorlinks=true,
        linkcolor=black,
        filecolor=magenta,
        urlcolor=blue,
        pdftitle={Markdown Converted to LaTeX},
        pdfpagemode=FullScreen,
        }
    \urlstyle{same}
    \renewcommand{\UrlFont}{\normalfont\itshape\underline}
    \usepackage[dvipsnames]{xcolor}

% Package for handling block quotes
    \usepackage{csquotes}

\begin{document}

% This removes page numbers from every page,
% Markdown doesn't use them so we don't need them!
    \pagestyle{empty}

\section{Grocery Shopping Guide}


Welcome to the \textbf{Ultimate Grocery Shopping Guide}! In this guide, we'll cover everything you need to know about shopping for groceries efficiently and effectively.


\subsection{Table of Contents}


\begin{enumerate}
    \item Planning Your Shopping Trip
    \item Selecting Fresh Produce
    \item Buying in Bulk
    \item Understanding Food Labels
    \item Budgeting for Groceries
    \item Storing Groceries
\end{enumerate}


\_\hrulefill


\subsubsection{Planning Your Shopping Trip}


Planning your trip is crucial for a successful grocery shopping experience.


\begin{itemize}
    \item  \textbf{Make a List:} Always start with a \textit{shopping list}. This helps in avoiding impulse buys.
    \item  \textbf{Hit the Right Store:} Finding a grocery store that fits your dietary needs is \textbf{crucial}. For example, \textit{Trader Joe's} offers many organic options.
    \item  \textbf{Stay Loyal to your store:} If your grocery store of choice offers a loyalty program, take them up on it! For example, \textit{Harris Teeter} has a program which offers a 10 cent per gallon discount on fuel for every \$100 spent up to \$1 off.
    \item  \textbf{Check for Deals:} Look for \textit{\textbf{coupons}} and discounts before you head out.
\end{itemize}


\enquote{\textbf{Tip:} Plan your meals for the week and make a list accordingly.}


\paragraph{Budgeting for Groceries}


\begin{itemize}
    \item Set a budget to control your spending.
    \item Compare prices online.
\end{itemize}


\paragraph{Selecting Fresh Produce}


\begin{enumerate}
    \item Look for fresh fruits and vegetables.
    \item Check for ripeness and quality.
\begin{itemize}
    \item You can tell whether a pineapple is ripe by sniffing the bottom!
\end{itemize}
\end{enumerate}


\_\hrulefill


\subsubsection{Buying in Bulk}


\textbf{Buying in bulk} can save money, but be cautious; it is easy to overbuy and then end up wasting food.


\begin{itemize}
    \item Don't buy perishables in bulk unless you're sure you can use them.
\end{itemize}


\paragraph{Understanding Food Labels}


Reading food labels is important for health-conscious shopping.

Understanding what the labels mean is just as important too!


\begin{verbatim}
Check the expiration date and nutritional information.
\end{verbatim}


\_\hrulefill


\subsubsection{Storing Groceries}


Proper storage extends the shelf life of your groceries.


\begin{itemize}
    \item Refrigerate perishables immediately.
    \item Store grains and spices in airtight containers.
\end{itemize}


\paragraph{Useful Links}


\begin{itemize}
    \item \href{https://www.healthyeating.org/}{Healthy Eating}
    \item \href{https://www.budgeting101$\backslash$.com/}{Budgeting Tips}
\end{itemize}


\_\hrulefill


\subsubsection{Example Shopping List}


\begin{itemize}
    \item Fruits
\begin{enumerate}
    \item Apples
    \item Oranges
\end{enumerate}
    \item Dairy
\begin{enumerate}
    \item Milk
    \item Cheese
\end{enumerate}
    \item Bakery
\begin{enumerate}
    \item Bread
    \item Bagels
\end{enumerate}
\end{itemize}


\_\hrulefill


\subsubsection{Showing off other Features}

\begin{verbatim}
Not grocery related sorry :(
\end{verbatim}


Multiline code looks like that and inline code looks like \texttt{this}!

\end{document}